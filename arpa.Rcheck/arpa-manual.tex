\nonstopmode{}
\documentclass[letterpaper]{book}
\usepackage[times,inconsolata,hyper]{Rd}
\usepackage{makeidx}
\usepackage[utf8,latin1]{inputenc}
% \usepackage{graphicx} % @USE GRAPHICX@
\makeindex{}
\begin{document}
\chapter*{}
\begin{center}
{\textbf{\huge Package `arpa'}}
\par\bigskip{\large \today}
\end{center}
\begin{description}
\raggedright{}
\item[Type]\AsIs{Package}
\item[Title]\AsIs{Parses ARPA language model files}
\item[Version]\AsIs{0.1}
\item[Date]\AsIs{2015-07-18}
\item[Author]\AsIs{Adam Acosta}
\item[Maintainer]\AsIs{Adam Acosta }\email{adam.acosta@gatech.edu}\AsIs{}
\item[Description]\AsIs{Parse language models read from ARPA files to R objects.}
\item[Depends]\AsIs{hash, stringi}
\item[Suggests]\AsIs{testthat}
\item[License]\AsIs{MIT}
\item[URL]\AsIs{}\url{https://github.com/adamacosta/arpa/wiki}\AsIs{}
\item[NeedsCompilation]\AsIs{no}
\end{description}
\Rdcontents{\R{} topics documented:}
\inputencoding{utf8}
\HeaderA{bigrams,ngram.model-method}{Return the bigrams of a language model}{bigrams,ngram.model.Rdash.method}
%
\begin{Description}\relax
Return the bigrams of a language model
\end{Description}
%
\begin{Usage}
\begin{verbatim}
## S4 method for signature 'ngram.model'
bigrams(object)
\end{verbatim}
\end{Usage}
%
\begin{Arguments}
\begin{ldescription}
\item[\code{object}] An ngram.model object
\end{ldescription}
\end{Arguments}
%
\begin{Value}
bigrams The bigrams of the model
\end{Value}
%
\begin{Author}\relax
Adam Acosta
\end{Author}
\inputencoding{utf8}
\HeaderA{contains,ngram.model,character-method}{Tests whether or not a language model contains an ngram}{contains,ngram.model,character.Rdash.method}
%
\begin{Description}\relax
Tests whether or not a language model contains an ngram
\end{Description}
%
\begin{Usage}
\begin{verbatim}
## S4 method for signature 'ngram.model,character'
contains(object, key)
\end{verbatim}
\end{Usage}
%
\begin{Arguments}
\begin{ldescription}
\item[\code{object}] An ngram.model object

\item[\code{key}] A character string
\end{ldescription}
\end{Arguments}
%
\begin{Value}
boolean Whether or not the string is in the language model
\end{Value}
%
\begin{Author}\relax
Adam Acosta
\end{Author}
\inputencoding{utf8}
\HeaderA{ngram.model-class}{An S4 class to represent an ngram language model}{ngram.model.Rdash.class}
\aliasA{ngram.model}{ngram.model-class}{ngram.model}
%
\begin{Description}\relax
Provides an efficient, fast way to map ngrams in a language
model to their log probability, allowing for easy next-word prediction.
\end{Description}
%
\begin{Section}{Slots}

\begin{description}

\item[\code{unigrams}] A hash table mapping unigrams to their log probability

\item[\code{bigrams}] A hash table mapping bigrams to their log probability

\item[\code{trigrams}] A hash table mapping trigrams to their log probability

\end{description}
\end{Section}
%
\begin{Author}\relax
Adam Acosta
\end{Author}
\inputencoding{utf8}
\HeaderA{read.arpa}{Read ARPA file}{read.arpa}
%
\begin{Description}\relax
Reads an ARPA file and returns the language model. See
http://www.speech.sri.com/projects/srilm/manpages/ngram-format.5.html
for a description of the ARPA file format.
\end{Description}
%
\begin{Usage}
\begin{verbatim}
read.arpa(input = "", header = TRUE, verbose = FALSE, nrow = -1L,
  skip = 0L, ugrams = -1L, bgrams = -1L, tgrams = -1L)
\end{verbatim}
\end{Usage}
%
\begin{Arguments}
\begin{ldescription}
\item[\code{input}] A filename

\item[\code{header}] boolean indicating whether or not the file has a header.
Default is TRUE

\item[\code{verbose}] A boolean indicating whether you want the function to
print information to the console as it is parsing the file. Default is FALSE.

\item[\code{nrow}] The number of rows in the file. This is optional, as the format
itself dictates that the file header must give the number of each ngram,
from which the number of lines in the file can be inferred. If this is passed
as a parameter, pass the number of unigrams, bigrams, and trigrams as well,
and it will speed up the parsing process.

\item[\code{skip}] The number of rows, if any, to skip in the file. Default is 0.

\item[\code{ugrams}] See below

\item[\code{bgrams}] See below

\item[\code{tgrams}] integer values indicating the number of unigrams, bigrams,
and trigrams. Should be in the header but can be passed to the function
directly to avoid header parsing overhead.
\end{ldescription}
\end{Arguments}
%
\begin{Value}
An ngram.model object, stored internally as a list of three hash
tables mapping each ngram to its log probability.
\end{Value}
%
\begin{Author}\relax
Adam Acosta
\end{Author}
\inputencoding{utf8}
\HeaderA{trigrams,ngram.model-method}{Return the trigrams of a language model}{trigrams,ngram.model.Rdash.method}
%
\begin{Description}\relax
Return the trigrams of a language model
\end{Description}
%
\begin{Usage}
\begin{verbatim}
## S4 method for signature 'ngram.model'
trigrams(object)
\end{verbatim}
\end{Usage}
%
\begin{Arguments}
\begin{ldescription}
\item[\code{object}] An ngram.model object
\end{ldescription}
\end{Arguments}
%
\begin{Value}
trigrams The trigrams of the model
\end{Value}
%
\begin{Author}\relax
Adam Acosta
\end{Author}
\inputencoding{utf8}
\HeaderA{unigrams,ngram.model-method}{Return the unigrams of a language model}{unigrams,ngram.model.Rdash.method}
%
\begin{Description}\relax
Return the unigrams of a language model
\end{Description}
%
\begin{Usage}
\begin{verbatim}
## S4 method for signature 'ngram.model'
unigrams(object)
\end{verbatim}
\end{Usage}
%
\begin{Arguments}
\begin{ldescription}
\item[\code{object}] An ngram.model object
\end{ldescription}
\end{Arguments}
%
\begin{Value}
unigrams The unigrams of the model
\end{Value}
%
\begin{Author}\relax
Adam Acosta
\end{Author}
\printindex{}
\end{document}
